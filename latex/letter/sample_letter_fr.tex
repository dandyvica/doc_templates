% pragma-like magic comments
% !TEX program = xelatex

% define a new A4 document with 12pt font size
\documentclass[a4paper,12pt]{article}

% for mobile phone etc symbols
\usepackage{marvosym}
	
% french accents etc	
\usepackage[francais]{babel}

% set margins with no header and no footer
\usepackage[left=1.9cm,top=2cm,bottom=2.5cm,right=1.9cm,nohead,nofoot]{geometry}

% if fancy tables are needed
\usepackage{array}

% if colors are needed
\usepackage{color}

% to use TrueType fonts. Caveat: need to compile with XeLatex
\usepackage{fontspec}
\setmainfont{Linux Libertine O}

% no page numbers
\pagestyle{empty}

% for images and set image files list of paths
\usepackage{graphicx}
\graphicspath{{.}{../../../Photos/}}

% no paragraph identation (or change it to whatever is needed)
\setlength{\parindent}{0pt}

% space between paragraphs
\setlength{\parskip}{11pt}

% interline space
\linespread{1.125}

% PDF metadata
\usepackage[
            pdfauthor={Victor Hugo},
			pdftitle={Les Misérables},
			xetex
            ]{hyperref}
            
% specific to French language
\frenchspacing

%% ------------------------------------------------------------
%% document start
%% ------------------------------------------------------------
\begin{document}

%% ------------------------------------------------------------
%% header table
%% ------------------------------------------------------------
\begin{tabular}{p{10cm} p{5cm}}
	
	\begin{minipage}[t]{0.40\textwidth}
		\begin{flushleft}
			{\Large \textbf{Victor Hugo}}\\[10pt]
			11 rue de la Liberté \\
			75001 Paris\\[8pt]
			\Mobilefone ~ +33 06 00 00 00 00\\
			\Email ~ victor.hugo@outlook.fr\\[20pt]
		\end{flushleft}
	\end{minipage}
	
	&
	
	\begin{minipage}[t]{0.35\textwidth}
		\begin{flushright}
			{\Large \textbf{Société Bertrand}}\\[10pt]
			11 rue de Rivoli\\
			75001 Paris\\
		\end{flushright}
	\end{minipage}  
	
\end{tabular}



\begin{flushleft}
Location, and date here

Letter object here
\end{flushleft}



\vspace{1.5cm}

%% ------------------------------------------------------------
%% Letter body
%% ------------------------------------------------------------
Monsieur Myriel,

\vspace{1.5cm}

En 1815, M. Charles-François-Bienvenu Myriel était évêque de Digne.
C'était un vieillard d'environ soixante-quinze ans; il occupait le siège
de Digne depuis 1806.

Quoique ce détail ne touche en aucune manière au fond même de ce que
nous avons à raconter, il n'est peut-être pas inutile, ne fût-ce que
pour être exact en tout, d'indiquer ici les bruits et les propos qui
avaient couru sur son compte au moment où il était arrivé dans le
diocèse. Vrai ou faux, ce qu'on dit des hommes tient souvent autant de
place dans leur vie et surtout dans leur destinée que ce qu'ils font. M.
Myriel était fils d'un conseiller au parlement d'Aix; noblesse de robe.

On contait de lui que son père, le réservant pour hériter de sa charge,
l'avait marié de fort bonne heure, à dix-huit ou vingt ans, suivant un
usage assez répandu dans les familles parlementaires. Charles Myriel,
nonobstant ce mariage, avait, disait-on, beaucoup fait parler de lui. Il
était bien fait de sa personne, quoique d'assez petite taille, élégant,
gracieux, spirituel; toute la première partie de sa vie avait été donnée
au monde et aux galanteries.

 La révolution survint, les événements se
précipitèrent, les familles parlementaires décimées, chassées, traquées,
se dispersèrent. M. Charles Myriel, dès les premiers jours de la
révolution, émigra en Italie. Sa femme y mourut d'une maladie de
poitrine dont elle était atteinte depuis longtemps. Ils n'avaient point
d'enfants. Que se passa-t-il ensuite dans la destinée de M. Myriel?


\vspace{1cm}
\hspace{10cm} Victor Hugo

\end{document}